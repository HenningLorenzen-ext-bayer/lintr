\documentclass[10pt,a4paper,onecolumn]{article}
\usepackage{marginnote}
\usepackage{graphicx}
\usepackage{xcolor}
\usepackage{authblk,etoolbox}
\usepackage{titlesec}
\usepackage{calc}
\usepackage{tikz}
\usepackage{hyperref}
\hypersetup{colorlinks,breaklinks,
            urlcolor=[rgb]{0.0, 0.5, 1.0},
            linkcolor=[rgb]{0.0, 0.5, 1.0}}
\usepackage{caption}
\usepackage{tcolorbox}
\usepackage{amssymb,amsmath}
\usepackage{ifxetex,ifluatex}
\usepackage{seqsplit}
\usepackage{fixltx2e} % provides \textsubscript
\usepackage[
  backend=biber,
%  style=alphabetic,
%  citestyle=numeric
]{biblatex}
\bibliography{paper.bib}


% --- Page layout -------------------------------------------------------------
\usepackage[top=3.5cm, bottom=3cm, right=1.5cm, left=1.0cm,
            headheight=2.2cm, reversemp, includemp, marginparwidth=4.5cm]{geometry}

% --- Default font ------------------------------------------------------------
% \renewcommand\familydefault{\sfdefault}

% --- Style -------------------------------------------------------------------
\renewcommand{\bibfont}{\small \sffamily}
\renewcommand{\captionfont}{\small\sffamily}
\renewcommand{\captionlabelfont}{\bfseries}

% --- Section/SubSection/SubSubSection ----------------------------------------
\titleformat{\section}
  {\normalfont\sffamily\Large\bfseries}
  {}{0pt}{}
\titleformat{\subsection}
  {\normalfont\sffamily\large\bfseries}
  {}{0pt}{}
\titleformat{\subsubsection}
  {\normalfont\sffamily\bfseries}
  {}{0pt}{}
\titleformat*{\paragraph}
  {\sffamily\normalsize}


% --- Header / Footer ---------------------------------------------------------
\usepackage{fancyhdr}
\pagestyle{fancy}
\fancyhf{}
%\renewcommand{\headrulewidth}{0.50pt}
\renewcommand{\headrulewidth}{0pt}
\fancyhead[L]{\hspace{-0.75cm}\includegraphics[width=5.5cm]{/Library/Frameworks/R.framework/Versions/4.2-arm64/Resources/library/rticles/rmarkdown/templates/joss/resources/JOSS-logo.png}}
\fancyhead[C]{}
\fancyhead[R]{}
\renewcommand{\footrulewidth}{0.25pt}

\fancyfoot[L]{\footnotesize{\sffamily , (). Static Code Analysis for
R. \textit{Journal of Open Source Software}, (), . \href{https://doi.org/}{https://doi.org/}}}


\fancyfoot[R]{\sffamily \thepage}
\makeatletter
\let\ps@plain\ps@fancy
\fancyheadoffset[L]{4.5cm}
\fancyfootoffset[L]{4.5cm}

% --- Macros ---------

\definecolor{linky}{rgb}{0.0, 0.5, 1.0}

\newtcolorbox{repobox}
   {colback=red, colframe=red!75!black,
     boxrule=0.5pt, arc=2pt, left=6pt, right=6pt, top=3pt, bottom=3pt}

\newcommand{\ExternalLink}{%
   \tikz[x=1.2ex, y=1.2ex, baseline=-0.05ex]{%
       \begin{scope}[x=1ex, y=1ex]
           \clip (-0.1,-0.1)
               --++ (-0, 1.2)
               --++ (0.6, 0)
               --++ (0, -0.6)
               --++ (0.6, 0)
               --++ (0, -1);
           \path[draw,
               line width = 0.5,
               rounded corners=0.5]
               (0,0) rectangle (1,1);
       \end{scope}
       \path[draw, line width = 0.5] (0.5, 0.5)
           -- (1, 1);
       \path[draw, line width = 0.5] (0.6, 1)
           -- (1, 1) -- (1, 0.6);
       }
   }

% --- Title / Authors ---------------------------------------------------------
% patch \maketitle so that it doesn't center
\patchcmd{\@maketitle}{center}{flushleft}{}{}
\patchcmd{\@maketitle}{center}{flushleft}{}{}
% patch \maketitle so that the font size for the title is normal
\patchcmd{\@maketitle}{\LARGE}{\LARGE\sffamily}{}{}
% patch the patch by authblk so that the author block is flush left
\def\maketitle{{%
  \renewenvironment{tabular}[2][]
    {\begin{flushleft}}
    {\end{flushleft}}
  \AB@maketitle}}
\makeatletter
\renewcommand\AB@affilsepx{ \protect\Affilfont}
%\renewcommand\AB@affilnote[1]{{\bfseries #1}\hspace{2pt}}
\renewcommand\AB@affilnote[1]{{\bfseries #1}\hspace{3pt}}
\makeatother
\renewcommand\Authfont{\sffamily\bfseries}
\renewcommand\Affilfont{\sffamily\small\mdseries}
\setlength{\affilsep}{1em}


\ifnum 0\ifxetex 1\fi\ifluatex 1\fi=0 % if pdftex
  \usepackage[T1]{fontenc}
  \usepackage[utf8]{inputenc}

\else % if luatex or xelatex
  \ifxetex
    \usepackage{mathspec}
  \else
    \usepackage{fontspec}
  \fi
  \defaultfontfeatures{Ligatures=TeX,Scale=MatchLowercase}

\fi
% use upquote if available, for straight quotes in verbatim environments
\IfFileExists{upquote.sty}{\usepackage{upquote}}{}
% use microtype if available
\IfFileExists{microtype.sty}{%
\usepackage{microtype}
\UseMicrotypeSet[protrusion]{basicmath} % disable protrusion for tt fonts
}{}

\usepackage{hyperref}
\hypersetup{unicode=true,
            pdftitle={Static Code Analysis for R},
            pdfborder={0 0 0},
            breaklinks=true}
\urlstyle{same}  % don't use monospace font for urls
\usepackage{graphicx,grffile}
\makeatletter
\def\maxwidth{\ifdim\Gin@nat@width>\linewidth\linewidth\else\Gin@nat@width\fi}
\def\maxheight{\ifdim\Gin@nat@height>\textheight\textheight\else\Gin@nat@height\fi}
\makeatother
% Scale images if necessary, so that they will not overflow the page
% margins by default, and it is still possible to overwrite the defaults
% using explicit options in \includegraphics[width, height, ...]{}
\setkeys{Gin}{width=\maxwidth,height=\maxheight,keepaspectratio}
\IfFileExists{parskip.sty}{%
\usepackage{parskip}
}{% else
\setlength{\parindent}{0pt}
\setlength{\parskip}{6pt plus 2pt minus 1pt}
}
\setlength{\emergencystretch}{3em}  % prevent overfull lines
\setcounter{secnumdepth}{0}
% Redefines (sub)paragraphs to behave more like sections
\ifx\paragraph\undefined\else
\let\oldparagraph\paragraph
\renewcommand{\paragraph}[1]{\oldparagraph{#1}\mbox{}}
\fi
\ifx\subparagraph\undefined\else
\let\oldsubparagraph\subparagraph
\renewcommand{\subparagraph}[1]{\oldsubparagraph{#1}\mbox{}}
\fi

% Pandoc syntax highlighting
\usepackage{color}
\usepackage{fancyvrb}
\newcommand{\VerbBar}{|}
\newcommand{\VERB}{\Verb[commandchars=\\\{\}]}
\DefineVerbatimEnvironment{Highlighting}{Verbatim}{commandchars=\\\{\}}
% Add ',fontsize=\small' for more characters per line
\usepackage{framed}
\definecolor{shadecolor}{RGB}{248,248,248}
\newenvironment{Shaded}{\begin{snugshade}}{\end{snugshade}}
\newcommand{\AlertTok}[1]{\textcolor[rgb]{0.94,0.16,0.16}{#1}}
\newcommand{\AnnotationTok}[1]{\textcolor[rgb]{0.56,0.35,0.01}{\textbf{\textit{#1}}}}
\newcommand{\AttributeTok}[1]{\textcolor[rgb]{0.77,0.63,0.00}{#1}}
\newcommand{\BaseNTok}[1]{\textcolor[rgb]{0.00,0.00,0.81}{#1}}
\newcommand{\BuiltInTok}[1]{#1}
\newcommand{\CharTok}[1]{\textcolor[rgb]{0.31,0.60,0.02}{#1}}
\newcommand{\CommentTok}[1]{\textcolor[rgb]{0.56,0.35,0.01}{\textit{#1}}}
\newcommand{\CommentVarTok}[1]{\textcolor[rgb]{0.56,0.35,0.01}{\textbf{\textit{#1}}}}
\newcommand{\ConstantTok}[1]{\textcolor[rgb]{0.00,0.00,0.00}{#1}}
\newcommand{\ControlFlowTok}[1]{\textcolor[rgb]{0.13,0.29,0.53}{\textbf{#1}}}
\newcommand{\DataTypeTok}[1]{\textcolor[rgb]{0.13,0.29,0.53}{#1}}
\newcommand{\DecValTok}[1]{\textcolor[rgb]{0.00,0.00,0.81}{#1}}
\newcommand{\DocumentationTok}[1]{\textcolor[rgb]{0.56,0.35,0.01}{\textbf{\textit{#1}}}}
\newcommand{\ErrorTok}[1]{\textcolor[rgb]{0.64,0.00,0.00}{\textbf{#1}}}
\newcommand{\ExtensionTok}[1]{#1}
\newcommand{\FloatTok}[1]{\textcolor[rgb]{0.00,0.00,0.81}{#1}}
\newcommand{\FunctionTok}[1]{\textcolor[rgb]{0.00,0.00,0.00}{#1}}
\newcommand{\ImportTok}[1]{#1}
\newcommand{\InformationTok}[1]{\textcolor[rgb]{0.56,0.35,0.01}{\textbf{\textit{#1}}}}
\newcommand{\KeywordTok}[1]{\textcolor[rgb]{0.13,0.29,0.53}{\textbf{#1}}}
\newcommand{\NormalTok}[1]{#1}
\newcommand{\OperatorTok}[1]{\textcolor[rgb]{0.81,0.36,0.00}{\textbf{#1}}}
\newcommand{\OtherTok}[1]{\textcolor[rgb]{0.56,0.35,0.01}{#1}}
\newcommand{\PreprocessorTok}[1]{\textcolor[rgb]{0.56,0.35,0.01}{\textit{#1}}}
\newcommand{\RegionMarkerTok}[1]{#1}
\newcommand{\SpecialCharTok}[1]{\textcolor[rgb]{0.00,0.00,0.00}{#1}}
\newcommand{\SpecialStringTok}[1]{\textcolor[rgb]{0.31,0.60,0.02}{#1}}
\newcommand{\StringTok}[1]{\textcolor[rgb]{0.31,0.60,0.02}{#1}}
\newcommand{\VariableTok}[1]{\textcolor[rgb]{0.00,0.00,0.00}{#1}}
\newcommand{\VerbatimStringTok}[1]{\textcolor[rgb]{0.31,0.60,0.02}{#1}}
\newcommand{\WarningTok}[1]{\textcolor[rgb]{0.56,0.35,0.01}{\textbf{\textit{#1}}}}

% tightlist command for lists without linebreak
\providecommand{\tightlist}{%
  \setlength{\itemsep}{0pt}\setlength{\parskip}{0pt}}


% Pandoc citation processing
\newlength{\cslhangindent}
\setlength{\cslhangindent}{1.5em}
\newlength{\csllabelwidth}
\setlength{\csllabelwidth}{3em}
\newlength{\cslentryspacingunit} % times entry-spacing
\setlength{\cslentryspacingunit}{\parskip}
% for Pandoc 2.8 to 2.10.1
\newenvironment{cslreferences}%
  {}%
  {\par}
% For Pandoc 2.11+
\newenvironment{CSLReferences}[2] % #1 hanging-ident, #2 entry spacing
 {% don't indent paragraphs
  \setlength{\parindent}{0pt}
  % turn on hanging indent if param 1 is 1
  \ifodd #1
  \let\oldpar\par
  \def\par{\hangindent=\cslhangindent\oldpar}
  \fi
  % set entry spacing
  \setlength{\parskip}{#2\cslentryspacingunit}
 }%
 {}
\usepackage{calc}
\newcommand{\CSLBlock}[1]{#1\hfill\break}
\newcommand{\CSLLeftMargin}[1]{\parbox[t]{\csllabelwidth}{#1}}
\newcommand{\CSLRightInline}[1]{\parbox[t]{\linewidth - \csllabelwidth}{#1}\break}
\newcommand{\CSLIndent}[1]{\hspace{\cslhangindent}#1}


\title{Static Code Analysis for R}

        \author[1]{Jim Hester}
    
      \affil[1]{Netflix}
  \date{\vspace{-5ex}}

\begin{document}
\maketitle

\marginpar{
  %\hrule
  \sffamily\small

  {\bfseries DOI:} \href{https://doi.org/}{\color{linky}{}}

  \vspace{2mm}

  {\bfseries Software}
  \begin{itemize}
    \setlength\itemsep{0em}
    \item \href{}{\color{linky}{Review}} \ExternalLink
    \item \href{}{\color{linky}{Repository}} \ExternalLink
    \item \href{}{\color{linky}{Archive}} \ExternalLink
  \end{itemize}

  \vspace{2mm}

  {\bfseries Submitted:} \\
  {\bfseries Published:} 

  \vspace{2mm}
  {\bfseries License}\\
  Authors of papers retain copyright and release the work under a Creative Commons Attribution 4.0 International License (\href{http://creativecommons.org/licenses/by/4.0/}{\color{linky}{CC-BY}}).
}

\hypertarget{statement-of-need}{%
\section{Statement of Need}\label{statement-of-need}}

The R programming language (\protect\hyperlink{ref-base2023}{R Core
Team, 2023}) is a popular choice for statistical analysis and
visualization, and is used by a wide range of researchers and data
scientists. The \texttt{\{lintr\}} package is an open-source R package
that provides static code analysis to check for a variety of common
problems related to readability, efficiency, consistency, style, etc. It
is designed to be easy to use and integrate into existing workflows, and
can be run from the command line or used as part of an automated build
or continuous integration process. \texttt{\{lintr\}} also integrates
with a number of popular IDEs and text editors, such as RStudio and
Visual Studio Code, making it convenient for users to run
\texttt{\{lintr\}} checks on their code as they work.

\hypertarget{features}{%
\section{Features}\label{features}}

There are over 85 linters offered by \texttt{\{lintr\}}!

\begin{Shaded}
\begin{Highlighting}[]
\FunctionTok{library}\NormalTok{(lintr)}

\FunctionTok{length}\NormalTok{(}\FunctionTok{all\_linters}\NormalTok{())}
\CommentTok{\#\textgreater{} [1] 87}
\end{Highlighting}
\end{Shaded}

Naturally, we can't discuss all of them here. To see details about all
available linters, we encourage readers to see
\url{https://lintr.r-lib.org/dev/reference/index.html\#individual-linters}.

We will showcase one linter for each kind of common problem found in R
code.

\begin{itemize}
\tightlist
\item
  \textbf{Best practices}
\end{itemize}

\texttt{\{lintr\}} offers linters that can detect problematic
antipatterns and suggest alternative patterns that follow best
practices.

For example, usage of vectorized \texttt{\&} and \texttt{\textbar{}}
logical operators in conditional statements is error-prone, and scalar
\texttt{\&\&} and \texttt{\textbar{}\textbar{}}, respectively, are to be
preferred. The \texttt{vector\_logic\_linter()} linter detects such
problematic usages.

\begin{Shaded}
\begin{Highlighting}[]
\FunctionTok{lint}\NormalTok{(}
  \AttributeTok{text =} \StringTok{"if (x \& y) 1"}\NormalTok{,}
  \AttributeTok{linters =} \FunctionTok{vector\_logic\_linter}\NormalTok{()}
\NormalTok{)}
\CommentTok{\#\textgreater{} \textless{}text\textgreater{}:1:7: warning: [vector\_logic\_linter] Conditional expressions require scalar logical operators (\&\& and ||)}
\CommentTok{\#\textgreater{} if (x \& y) 1}
\CommentTok{\#\textgreater{}       \^{}}
\end{Highlighting}
\end{Shaded}

\begin{itemize}
\tightlist
\item
  \textbf{Efficiency}
\end{itemize}

Sometimes the users might not be aware of a more efficient way offered
by R for carrying out a computation. \texttt{\{lintr\}} offers linters
to provide suggestions to improve code efficiency.

\begin{Shaded}
\begin{Highlighting}[]
\FunctionTok{lint}\NormalTok{(}
  \AttributeTok{text =} \StringTok{"any(is.na(x), na.rm = TRUE)"}\NormalTok{,}
  \AttributeTok{linters =} \FunctionTok{any\_is\_na\_linter}\NormalTok{()}
\NormalTok{)}
\CommentTok{\#\textgreater{} \textless{}text\textgreater{}:1:1: warning: [any\_is\_na\_linter] anyNA(x) is better than any(is.na(x)).}
\CommentTok{\#\textgreater{} any(is.na(x), na.rm = TRUE)}
\CommentTok{\#\textgreater{} \^{}\textasciitilde{}\textasciitilde{}\textasciitilde{}\textasciitilde{}\textasciitilde{}\textasciitilde{}\textasciitilde{}\textasciitilde{}\textasciitilde{}\textasciitilde{}\textasciitilde{}\textasciitilde{}\textasciitilde{}\textasciitilde{}\textasciitilde{}\textasciitilde{}\textasciitilde{}\textasciitilde{}\textasciitilde{}\textasciitilde{}\textasciitilde{}\textasciitilde{}\textasciitilde{}\textasciitilde{}\textasciitilde{}\textasciitilde{}}
\end{Highlighting}
\end{Shaded}

\begin{itemize}
\tightlist
\item
  \textbf{Readability}
\end{itemize}

Coders spend significantly more time reading compared to writing code
(\protect\hyperlink{ref-mcconnell2004code}{McConnell, 2004}). Thus,
writing readable code makes the code more maintainable and reduces the
possibility of introducing bugs stemming from a poor understanding of
the code.

\texttt{\{lintr\}} provides a number of linters that suggest more
readable alternatives. For example,
\texttt{function\_left\_parentheses\_linter()}.

\begin{Shaded}
\begin{Highlighting}[]
\FunctionTok{lint}\NormalTok{(}
  \AttributeTok{text =} \StringTok{"stats::sd (c (x, y, z))"}\NormalTok{,}
  \AttributeTok{linters =} \FunctionTok{function\_left\_parentheses\_linter}\NormalTok{()}
\NormalTok{)}
\CommentTok{\#\textgreater{} \textless{}text\textgreater{}:1:10: style: [function\_left\_parentheses\_linter] Remove spaces before the left parenthesis in a function call.}
\CommentTok{\#\textgreater{} stats::sd (c (x, y, z))}
\CommentTok{\#\textgreater{}          \^{}}
\CommentTok{\#\textgreater{} \textless{}text\textgreater{}:1:13: style: [function\_left\_parentheses\_linter] Remove spaces before the left parenthesis in a function call.}
\CommentTok{\#\textgreater{} stats::sd (c (x, y, z))}
\CommentTok{\#\textgreater{}             \^{}}
\end{Highlighting}
\end{Shaded}

\begin{itemize}
\tightlist
\item
  \textbf{Tidyverse style}
\end{itemize}

\texttt{\{lintr\}} also provides linters to enforce the style used
throughout the \texttt{\{tidyverse\}}
(\protect\hyperlink{ref-Wickham2019}{Wickham et al., 2019}) ecosystem of
R packages. This style of coding has been outlined in the tidyverse
style guide (\url{https://style.tidyverse.org/index.html}).

\begin{Shaded}
\begin{Highlighting}[]
\FunctionTok{lint}\NormalTok{(}
  \AttributeTok{text =} \StringTok{"1:3 \%\textgreater{}\% mean \%\textgreater{}\% as.character"}\NormalTok{,}
  \AttributeTok{linters =} \FunctionTok{pipe\_call\_linter}\NormalTok{()}
\NormalTok{)}
\CommentTok{\#\textgreater{} \textless{}text\textgreater{}:1:9: warning: [pipe\_call\_linter] Use explicit calls in magrittr pipes, i.e., \textasciigrave{}a \%\textgreater{}\% foo\textasciigrave{} should be \textasciigrave{}a \%\textgreater{}\% foo()\textasciigrave{}.}
\CommentTok{\#\textgreater{} 1:3 \%\textgreater{}\% mean \%\textgreater{}\% as.character}
\CommentTok{\#\textgreater{}         \^{}\textasciitilde{}\textasciitilde{}\textasciitilde{}}
\CommentTok{\#\textgreater{} \textless{}text\textgreater{}:1:18: warning: [pipe\_call\_linter] Use explicit calls in magrittr pipes, i.e., \textasciigrave{}a \%\textgreater{}\% foo\textasciigrave{} should be \textasciigrave{}a \%\textgreater{}\% foo()\textasciigrave{}.}
\CommentTok{\#\textgreater{} 1:3 \%\textgreater{}\% mean \%\textgreater{}\% as.character}
\CommentTok{\#\textgreater{}                  \^{}\textasciitilde{}\textasciitilde{}\textasciitilde{}\textasciitilde{}\textasciitilde{}\textasciitilde{}\textasciitilde{}\textasciitilde{}\textasciitilde{}\textasciitilde{}\textasciitilde{}}
\end{Highlighting}
\end{Shaded}

\hypertarget{benefits-of-using-lintr}{%
\section{\texorpdfstring{Benefits of using
\texttt{\{lintr\}}}{Benefits of using \{lintr\}}}\label{benefits-of-using-lintr}}

There are several benefits to using \texttt{\{lintr\}} to analyze and
improve R code. One of the most obvious is that it can help users
identify and fix problems in their code, which can save time and effort
during the development process. By catching issues early on,
\texttt{\{lintr\}} can help prevent bugs and other issues from creeping
into code, which can save time and effort when it comes to debugging and
testing.

Another benefit of \texttt{\{lintr\}} is that it can help users write
more readable and maintainable code. By enforcing a consistent style and
highlighting potential issues, \texttt{\{lintr\}} can help users write
code that is easier to understand and work with. This is especially
important for larger projects or teams, where multiple contributors may
be working on the same codebase and it is important to ensure that code
is easy to follow and understand.

\texttt{\{lintr\}} can be a useful tool for teaching and learning R. By
providing feedback on code style and potential issues, it can help users
learn good coding practices and improve their skills over time. This can
be especially useful for beginners, who may not yet be familiar with all
of the best practices for writing R code.

Finally, \texttt{\{lintr\}} has had a large and active user community
since its birth in 2014 which has contributed to its rapid development,
maintenance, and adoption. At the time of writing, \texttt{\{lintr\}} is
in a mature and stable state and therefore provides a reliable API that
is unlikely to feature any breaking changes.

\hypertarget{conclusion}{%
\section{Conclusion}\label{conclusion}}

In conclusion, \texttt{\{lintr\}} is a valuable tool for R users to help
improve the quality and reliability of their code. Its static code
analysis capabilities, combined with its flexibility and ease of use,
make it relevant and valuable for a wide range of applications.

\hypertarget{licensing-and-availability}{%
\section{Licensing and Availability}\label{licensing-and-availability}}

\texttt{\{lintr\}} is licensed under the MIT License, with all source
code openly developed and stored on GitHub
(\url{https://github.com/r-lib/lintr}), along with a corresponding issue
tracker for bug reporting and feature enhancements.

\hypertarget{acknowledgments}{%
\section{Acknowledgments}\label{acknowledgments}}

\hypertarget{references}{%
\section*{References}\label{references}}
\addcontentsline{toc}{section}{References}

\hypertarget{refs}{}
\begin{CSLReferences}{1}{0}
\leavevmode\vadjust pre{\hypertarget{ref-mcconnell2004code}{}}%
McConnell, S. (2004). \emph{Code complete}. Pearson Education.

\leavevmode\vadjust pre{\hypertarget{ref-base2023}{}}%
R Core Team. (2023). \emph{{R}: A language and environment for
statistical computing}. R Foundation for Statistical Computing.
\url{https://www.R-project.org/}

\leavevmode\vadjust pre{\hypertarget{ref-Wickham2019}{}}%
Wickham, H., Averick, M., Bryan, J., Chang, W., McGowan, L. D.,
François, R., Grolemund, G., Hayes, A., Henry, L., Hester, J., Kuhn, M.,
Pedersen, T. L., Miller, E., Bache, S. M., Müller, K., Ooms, J.,
Robinson, D., Seidel, D. P., Spinu, V., \ldots{} Yutani, H. (2019).
Welcome to the {tidyverse}. \emph{Journal of Open Source Software},
\emph{4}(43), 1686. \url{https://doi.org/10.21105/joss.01686}

\end{CSLReferences}

\end{document}
